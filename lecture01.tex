\chapter{Алгебра логики}

\section{Высказывания}

Перед тем, как доказывать всевозможные полезные в повседневной жизни
программиста факты и утверждения, необходимо договориться о некотором общем
языке, в рамках которого производится математика. В нашем случае, это язык
булевой логики, в которой основным объектом является \textit{высказывание}.

\begin{defn}
  \textit{Высказывание} --- предложение, о котором можно однозначно сказать,
  истинно оно или ложно.
\end{defn}

\begin{ex}
  ``В первые 4 миллиарда лет своего существования Земля вращалась вокруг
  Солнца'' --- истинное высказывание.
\end{ex}
\begin{ex}\label{sun}
  ``Сейчас на улице солнечно, и завтра у нас нет пар'' является высказыванием,
  хоть его истинность и зависит от текущего положения дел.
\end{ex}
\begin{ex}
  ``Либо работай, либо готовься к экзамену'' высказыванием не является,
  поскольку вопрос об истинности этого предложения не имеет смысла.
\end{ex}

Но каким образом мы можем рассуждать об истинности или ложности высказываний?
Заметим, что на самом деле высказывания бывают как минимум двух разных видов:

\begin{defn}
  \textit{Составное высказывание} --- высказывание, составленное из других,
  более простых, высказываний.
\end{defn}
\begin{defn}
  \textit{Атомарное высказывание} --- самое короткое, <<неделимое>>
  высказывание; то, которое нельзя разделить на более простые.
\end{defn}

Истинность атомарных высказываний математически установить не так просто; знание
о них дано нам заранее, в виде явного указания, истинны они или нет. Для
краткости записи, в алгебре логики каждое атомарное высказывание обозначают
отдельной латинской буквой --- $p$, $q$, $r$, $A$, $B$, $C$,\dots

В свою очередь, истинность составного высказывания уже можно определить, если
известна истинность его частей. Например, в примере \ref{sun}, если сейчас на
улице действительно солнечно и действительно завтра нет пар, то всё высказывание
целиком также истинно. Заметим, что мы смогли сделать этот вывод благодаря
интуитивному пониманию значения слова ``и''; в формальной логике, ему
соответствует \textit{логическая связка} ``И'' (также обозначается как $\land$).

\begin{defn}
  \textit{Логическая связка} --- конструкция, позволяющая получать новые
  составные высказывания из других (возможно, тоже составных) высказываний.
\end{defn}
\begin{ex}
  ``ИЛИ'' (также обозначается как $\lor$) является логической связкой, создающей
  новое высказывание из двух других.
\end{ex}
\begin{ex}
  ``НЕ'' (также обозначается как $\neg$) является логической связкой, создающей
  по высказыванию новое, являющееся противоположным по смыслу исходному.
\end{ex}

При желании, можно придумать свои, совершенно другие логические связки; что
важно, так это то, что же они всё-таки обозначают --- то есть, как истинность
получающегося при использовании связки утверждения зависит от истинностей
составных частей. Это можно задать с помощью \textit{таблицы истинности}.

\begin{defn}
  \textit{Таблица истинности} для связки $n$ высказываний --- таблица из $n+1$
  столбца и $2^n$ строчек, где для каждого возможного случая (набора из $n$
  фактов об истинности либо ложности каждой части составного высказывания)
  найдётся ровно одна строчка, где в первых $n$ клетках будет выписан этот
  случай, а в последней клетке указано, истинно ли соответствующее составное
  высказывание в этом случае.
\end{defn}

\begin{ex}
  Таблицы истинности для всех упомянутых логических связок (для краткости
  совмещены в одну таблицу):
  \begin{center}
  \begin{tabular}{cc|c|c|c}
    $p$ & $q$ & $p\land q$ & $p\lor q$ & $\neg p$ \\
    \hline
    0 & 0 & 0 & 0 & 1 \\
    0 & 1 & 0 & 1 & 1 \\
    1 & 0 & 0 & 1 & 0 \\
    1 & 1 & 1 & 1 & 0
  \end{tabular}
  \end{center}
\end{ex}

Кроме этого, есть ещё одна очень важная связка --- импликация $(\to)$.
\begin{center}
\begin{tabular}{cc|c}
  $p$&$q$&$p\to q$\\
  \hline
  0&0&1\\
  0&1&1\\
  1&0&0\\
  1&1&1
\end{tabular}
\end{center}

По смыслу, она соответствует утверждению формата ``если $p$, то $q$''. На первый
взгляд, её таблица истинности может показаться немного странной; однако же
рассмотрим следующие случаи:
\begin{itemize}
  \item Если $p$ ложно, то, каким бы ни было $q$, само по себе утверждение
    $p\to q$ истинно: из лжи может следовать что угодно.
  \item Если $q$ истинно, то $p \to q$ само по себе истинно уже хотя бы потому,
    что истинно $q$: в ``доказательстве'' для $p \to q$ мы можем воспользоваться
    ``доказательством'' для $q$, само по себе $p$ никак не затронув.
  \item Если $p$ истинно, а $q$ ложно, то $p \to q$ истинным быть не может по
    определению: из истинных утверждений не могут следовать ложные.
\end{itemize}

Кроме следствия одних утверждений из других, естественно задуматься о
равносильности утверждений: $(p\equiv q):=(p\to q)\land(q\to p)$ (здесь $:=$
обозначает ``равно по определению'').

Как уже говорилось ранее, в конечном итоге истинность абсолютного большинства
высказываний зависит от истинности атомарных высказываний, в них входящих; тем
не менее, есть некоторые особые случаи, которые мы рассмотрим далее.

\section{Эквивалентные высказывания и тавтологии}

\begin{defn}
  \textit{Эквивалентные высказывания} --- те высказывания, которые истинны
  одновременно (либо ложны одновременно) во всех возможных случаях.
\end{defn}

\begin{ex}
  Произвольное высказывание $A$ эквивалентно само себе: $A\Leftrightarrow A$.
\end{ex}

Проведём небольшое наблюдение: если $A\Leftrightarrow B$, то $A\equiv B$ истинно
всегда.

\begin{defn}
  Утверждения, истинные во всех возможных случаях (истинностях/ложностях
  входящих в них атомарных высказываний), называются \textit{тавтологиями}.
\end{defn}

\begin{defn}
  Утверждения, ложные во всех возможных случаях, называются
  \textit{противоречием}.
\end{defn}

Как доказывать эквивалентности и то, что утверждения являются тавтологиями?
Например, таблицами истинности.

\begin{ex}
  Следующие утверждения являются тавтологиями:
  \begin{tasks}(2)
    \task $p\land p\equiv p$;
    \task $p\lor p\equiv p$;
    \task $\neg(p\lor q)\equiv\neg p\land\neg q$;
    \task $\neg(p\land q)\equiv\neg p\lor\neg q$;
    \task $p\land q\equiv q\land p$;
    \task $p\lor q\equiv q\lor p$;
    \task $p\land(q\land r)\equiv(p\land q)\land r$;
    \task $p\lor(q\lor r)\equiv(p\lor q)\lor r$;
    \task $p\land(q\lor r)\equiv p\land q\lor p\land r$;
    \task $p\lor q\land r\equiv (p\lor q)\land(p\lor r)$;
    \task $\neg q\to\neg p\equiv p\to q$;
    \task $\neg(\neg p)\equiv p$;
  \end{tasks}
  (У $\neg$ самый высокий приоритет; $\land$ приоритет более высокий, чем у
  $\land$; у $\to$ --- ниже всех, кроме $\equiv$, имеющего самый низкий
  приоритет.)
\end{ex}

Кроме этого, существуют другие, более удобные способы доказательств
тавтологичности:
\begin{itemize}
  \item С помощью разбора случаев;
  \item С помощью замены частей высказывания на эквивалентные им;
  \item От противного.
\end{itemize}

\section{Кванторы существования и всеобщности}

На самом деле, в основаниях математики используется ещё чуть более сложный язык,
чем состоящий из атомарных высказываний и логических связок. В абсолютном
большинстве случаев перед математиком --- да и любым исследователем,
использующим математику как инструмент, в том числе и грамотным программистом
--- стоит задача не просто установления истинности либо ложности некоторых
утверждений, но исследования свойств некоторого математического объекта.
Для этого необходимы выразительные средства, позволяющие рассуждать о
совокупностях объектов.

Зачастую объекты, находящиеся на рассмотрении математика, обозначаются
латинскими буквами: $x$, $y$, $z$, $G$, $V$, $E$,\dots. Далее мы можем
формулировать утверждения формата ``$x$ делится на 15 без остатка'' и т.д.

Но откуда берутся все эти иксы и игреки, о которых мы говорим? Существует два
основных способа рассуждать об объектах:

\begin{itemize}
  \item как о некотором достоверно существующем объекте, который мы просто
    обозначаем буквой;
  \item как о произвольном по своей сути объекте --- в качестве обозначаемого
    должен подойти любой из них.
\end{itemize}

Для каждого из них есть своё средство в языке, называемое \textit{квантором}.

\textit{Квантор существования} $\exists x.\varphi(x)$ позволяет формулировать
высказывания вида ``существует $x$ такой, что выполнено $\varphi(x)$'', где
$\varphi(x)$ --- произвольное высказывание про $x$.

\begin{ex}
  Высказывание ``единорогов не существует'' можно записать как
  ``$\neg\exists x.x\textrm{ --- единорог}$'' (на всякий случай замечу, что
  истинности этого высказывания я не утверждал).
\end{ex}

\textit{Квантор всеобщности} $\forall x.\varphi(x)$ позволяет формулировать
высказывания вида ``для любого $x$ выполнено $\varphi(x)$'', где $\varphi(x)$
--- произвольное высказывание про $x$.

\begin{ex}
  Высказывание ``не существует максимального простого числа'' можно записать как
  ``$\neg\exists x.(x\textrm{ --- простое})\land
  \forall y.(y\textrm{ --- простое})\to y\leqslant x$''.
\end{ex}

При попытке определить истинность высказываний с, например, квантором
всеобщности $\forall x$ возникает вопрос: из какой совокупности берётся $x$? Это
произвольный математический объект? Объект реального мира? Формула? Число?

В действительности, при использовании формул с кванторами всегда заранее
фиксируется, из какого набора подбираются объекты. Этот набор называется
\textit{носителем}, либо \textit{предметной областью}; строго говоря, носитель
должен являться множеством. А что это значит, мы рассмотрим в следующей главе.
